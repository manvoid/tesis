%*******************************************************
% Abstract
%*******************************************************
%\renewcommand{\abstractname}{Abstract}
\pdfbookmark[1]{Abstract}{Abstract}
\begingroup
\let\clearpage\relax
\let\cleardoublepage\relax
\let\cleardoublepage\relax

\chapter*{Abstract}
Las interfaces computacionales son la primer capa que se encuentra entre el usuario y el sistema, en los últimos años se han hecho grandes avances en como es que estas interfaces ayudan a los usuarios a entender de una manera más sencilla y directa el sistema además de permitirle obtener el mayor provecho posible de este. Sin embargo las interfaces en robótica no han tenido demasiado desarrollo, al menos en la robótica a nivel licenciatura, pero es importante que desde un inicio se piense como es que se pueden utilizar interfaces robustas e intuitivas para mejorar el uso de los sistemas en general. En este trabajo se propone el uso de las tecnologías web, las cuales han sido ampliamente desarrolladas en los últimos años, como una alternativa para crear interfaces modernas en diversos proyectos de robótica\dots


\vfill

\pdfbookmark[1]{Resúmen}{Resúmen}
\chapter*{Resúmen}
Las interfaces computacionales son la primer capa que se encuentra entre el usuario y el sistema, en los últimos años se han hecho grandes avances en como es que estas interfaces ayudan a los usuarios a entender de una manera más sencilla y directa el sistema además de permitirle obtener el mayor provecho posible de este. Sin embargo las interfaces en robótica no han tenido demasiado desarrollo, al menos en la robótica a nivel licenciatura, pero es importante que desde un inicio se piense como es que se pueden utilizar interfaces robustas e intuitivas para mejorar el uso de los sistemas en general. En este trabajo se propone el uso de las tecnologías web, las cuales han sido ampliamente desarrolladas en los últimos años, como una alternativa para crear interfaces modernas en diversos proyectos de robótica\dots


\endgroup			

\vfill